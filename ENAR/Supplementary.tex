\documentclass[fleqn,12pt]{article}

\usepackage{mycommands,amssymb,amsmath,amsthm,color,pagesize,outlines,cite,subfigure}
\usepackage[small]{caption}
\usepackage[pdftex]{epsfig}
\usepackage{hyperref} % for linking references 
\usepackage{stackrel}
\usepackage{rotating} % sideways table

\usepackage[round]{natbib}

% for algorithm
\usepackage[noend]{algpseudocode}
\usepackage{algorithm}

%% measurements for 1 inch margin
\addtolength{\oddsidemargin}{-.875in}
\addtolength{\evensidemargin}{-.875in}
\addtolength{\textwidth}{1.75in}
\addtolength{\topmargin}{-.875in}
\addtolength{\textheight}{1.75in}

\usepackage{setspace}
\doublespacing

\usepackage{chngcntr}
\usepackage{apptools}
\AtAppendix{\counterwithin{Theorem}{section}}
\numberwithin{equation}{section}

\begin{document}

\newtheorem{Theorem}{Theorem}[section]
\newtheorem{Lemma}[Theorem]{Lemma}
\newtheorem{Corollary}[Theorem]{Corollary}
\newtheorem{Proposition}[Theorem]{Proposition}
\newtheorem{Conjecture}[Theorem]{Conjecture}
\theoremstyle{definition} \newtheorem{Definition}[Theorem]{Definition}

\title{Supplementary to "Simultaneous Selection of Multiple Important Single Nucleotide Polymorphisms in Familial Genome Wide Association Studies data"}
\date{}
\maketitle

\appendix

\section{Proof of theoretical results}

\begin{proof}[Proof of Theorem 3.2] Define $c_{q,\infty} = q^{\text{th}}$ quantile of $\BT_0$. Now following assumption (E1),
%
\begin{align*}
c_q (\BE_*) &= \inf_\bftheta \{ E(\bftheta, [\hat \bftheta]): \BF_* \geq q \} \\
&= \inf_\bftheta \{ E(a_n(\bftheta - \bftheta_0), [a_n(\hat \bftheta - \bftheta_0)]):  a_n (\BF_* - \bftheta_0) \geq q\}
\end{align*}
%
where $\BF_*$ is the probability distribution function of $E(\hat \bftheta, [\hat \bftheta])$. Part 1 is proved following assumptions (P2) and (E3).

Now if $\cM$ is adequate, following assumption (E1),
%
\begin{align}\label{eqn:Thm1ProofEqn1}
E (\hat \bftheta_m, [\hat \bftheta]) = E( \hat \bftheta	_m - \bftheta_0, [\hat \bftheta - \bftheta_0])
\end{align}
%
Decompose the first argument as
%
\begin{align}\label{eqn:Thm1ProofEqn2}
\hat \bftheta_m - \bftheta &= (\hat \bftheta_m - \hat \bftheta) + (\hat \bftheta - \bftheta_0)
\end{align}
%
By definition, $\hat \theta_{mj} - \hat \theta_j = 0$ if $j \in \cS$, else equals $\theta_{0j} - \hat \theta_j$. Thus for the first summand in (\ref{eqn:Thm1ProofEqn2}) we have
%
$$
\hat \bftheta_m - \hat \bftheta =  O_P (1/a_n)
$$
%
Going back to (\ref{eqn:Thm1ProofEqn1}), this implies
%
$$
| E( \hat \bftheta_m - \bftheta_0, [\hat \bftheta - \bftheta_0]) -
E( \hat \bftheta - \bftheta_0, [\hat \bftheta - \bftheta_0]) | < O_P (a_n^{-\alpha})
$$
%
using lipschitz continuity in assumption (E2), i.e
%
$$
| E( \hat \bftheta_m, [\hat \bftheta]) - E( \hat \bftheta, [\hat \bftheta]) | < O_P (a_n^{-\alpha})
$$
%
again using (E1). Part 2 now follows.

For part 3, we apply (E1) to get
%
\begin{align}\label{eqn:Thm1ProofEqn3}
E (\hat \bftheta_m, [\hat \bftheta]) = E( a_n(\hat \bftheta	_m - \bftheta_0), [a_n(\hat \bftheta - \bftheta_0)])
\end{align}
%
And decompose the first argument as
%
\begin{align}\label{eqn:Thm1ProofEqn4}
a_n(\hat \bftheta_m - \bftheta_0) &= a_n(\hat \bftheta_m - \bftheta_m) + a_n(\bftheta_m - \bftheta_0)
\end{align}
%
Since $\cM$ is inadequate, $\theta_{mj} \neq \theta_{0j}$ when $j \notin \cS$. So $\| a_n(\bftheta_m - \bftheta_0) \| \uparrow \infty$ as $a_n \uparrow \infty$. Applying (E4) now proves part 3.
\end{proof}

\begin{proof}[Proof of Theorem 3.3] The proof is fairly similar to that of theorem 3.2, so we give a sketch of it. For the full model, the bootstrap is consistent, i.e. $a_n (\hat \bftheta_* - \bftheta_0)$ and $(a_n/ \tau_n) (\hat \bftheta_{r*} - \hat \bftheta_*)$ converge to same weak limit in probability, following theorems 2.2 and 2.3 in \cite{MajumdarChatterjee17}. Specifically, conditions (A1)-(A6) in \cite{MajumdarChatterjee17} ensure condition (P2) in our paper through theorem 2.2 therein, following which  theorem 2.3 ensures that when (A1)-(A6) are satisfied, bootstrap consistency holds. The definition of $\hat \bftheta_m$ now means that
$a_n (\hat \bftheta_m - \bftheta_m)$ and $(a_n/ \tau_n) (\hat \bftheta_{rm} - \hat \bftheta_m)$ converge to the same weak limit in probability for any model $\cM$. A similar approach as the proof of parts 2 and 3 of theorem 3.2 now follows, with an additional term corresponding to bootstrap estimates in (\ref{eqn:Thm1ProofEqn2}) and (\ref{eqn:Thm1ProofEqn4}).
\end{proof}

\section{Outputs for MCTFR data analysis}

Each table gives the $90^{\text{th}}$ percentile $e$-values, which are plotted in figures 2, 3, and 4 in main paper, of SNPs analyzed in the gene. Column 'Association' is obtained from the sign of the SNP coefficient in the full model.

% latex table generated in R 3.3.2 by xtable 1.8-2 package
% Fri Oct 13 17:34:26 2017
\begin{table}[ht]
\centering
\begin{tabular}{lccc}
  \hline
SNP name & Location & $e$-value & Association \\ 
  \hline
rs16859227 & 46250605 & 0.89 & + \\ 
  rs572227 & 46251393 & 0.13 & - \\ 
  rs534459 & 46256805 & 0.24 & + \\ 
  rs2119183 & 46272806 & 0.92 & - \\ 
  rs502038 & 46280318 & 0.58 & + \\ 
  rs1808851 & 46311447 & 0.00 & + \\ 
  rs279856 & 46317923 & 0.00 & - \\ 
  rs3775282 & 46321863 & 0.86 & - \\ 
  rs279841 & 46340763 & 0.75 & + \\ 
  rs10805145 & 46358331 & 0.73 & - \\ 
  rs13152740 & 46381221 & 0.86 & - \\ 
   \hline
\end{tabular}
\caption{SNPs for GABRA2, chr4, position 46243548 - 46390039; $e$-value cutoff 0.72}
\end{table}

% latex table generated in R 3.3.2 by xtable 1.8-2 package
% Fri Oct 13 17:34:26 2017
\begin{table}[H]
\centering
\begin{tabular}{lccc}
  \hline
SNP name & Location & $e$-value & Association \\ 
  \hline
rs17027299 & 99078105 & 0.84 & - \\ 
  rs9307222 & 99101051 & 0.76 & - \\ 
  rs10006414 & 99101401 & 0.49 & + \\ 
  rs9994641 & 99101605 & 0.48 & + \\ 
  rs13134014 & 99104879 & 0.75 & - \\ 
  rs6820691 & 99105055 & 0.76 &+\\ 
  rs6820913 & 99125659 & 0.81 &+\\ 
  rs6532729 & 99146436 & 0.67 &-\\ 
  rs13150538 & 99152631 & 0.49 &-\\ 
  rs17027380 & 99157450 & 0.63 &-\\ 
  rs17494998 & 99160699 & 0.41 &+\\ 
  rs549467 & 99172232 & 0.81 &+\\ 
  rs2034677 & 99187874 & 0.62 &+\\ 
  rs12508445 & 99190653 & 0.01 &-\\ 
  rs10003496 & 99197839 & 0.81 &+\\ 
  rs10005811 & 99208603 & 0.02 &+\\ 
  rs603215 & 99214851 & 0.78 &-\\ 
  rs433146 & 99229839 & 0.87 &-\\ 
  rs17027456 & 99235747 & 0.31 &-\\ 
  rs17561798 & 99235941 & 0.85 &+\\ 
  rs10516428 & 99237439 & 0.45 &-\\ 
  rs6532731 & 99251006 & 0.80 &+\\ 
  rs7694221 & 99260423 & 0.90 &+\\ 
  rs10028330 & 99268949 & 0.70 &-\\ 
  rs10022047 & 99296818 & 0.49 &+\\ 
  rs17027523 & 99298979 & 0.05 &+\\ 
  rs17027530 & 99303633 & 0.69 &+\\ 
  rs3775540 & 99304544 & 0.23 &-\\ 
  rs3756088 & 99309404 & 0.89 &-\\ 
  rs13103626 & 99317251 & 0.75 &+\\ 
  rs10516430 & 99337881 & 0.62 &+\\ 
  rs9884594 & 99359318 & 0.68 &-\\ 
  rs12503056 & 99369061 & 0.63 &+\\ 
  rs2004316 & 99381148 & 0.43 &-\\ 
  rs4303985 & 99399748 & 0.87 &-\\ 
  rs4414961 & 99403784 & 0.86 &-\\ 
  rs12509267 & 99407299 & 0.80 &+\\ 
  rs6838913 & 99408106 & 0.84 &-\\ 
  rs4374629 & 99411783 & 0.85 &+\\ 
  rs4527483 & 99421741 & 0.89 &+\\ 
  rs10009693 & 99423280 & 0.90 &-\\ 
  rs10023791 & 99425353 & 0.88 &+\\ 
  rs955931 & 99428163 & 0.88 &-\\ 
  rs17027628 & 99428608 & 0.85 &-\\ 
   \hline
\end{tabular}
\caption{SNPs for ADH genes, chr4, position 99070977 - 99435737; $e$-value cutoff 0.225}
\end{table}

% latex table generated in R 3.3.2 by xtable 1.8-2 package
% Fri Oct 13 17:34:26 2017
\begin{table}[H]
\centering
\begin{tabular}{lccc}
  \hline
SNP name & Location & $e$-value & Association \\ 
  \hline
rs2000371 & 154011024 & 0.39 &-\\ 
  rs9371718 & 154011615 & 0.08 &-\\ 
  rs12211203 & 154016936 & 0.63 &-\\ 
  rs1937600 & 154017197 & 0.02 &-\\ 
  rs9397637 & 154022718 & 0.00 &+\\ 
  rs1937590 & 154036895 & 0.63 &+\\ 
  rs12662873 & 154040810 & 0.18 &+\\ 
  rs12661209 & 154044112 & 0.84 &-\\ 
  rs1316368 & 154055754 & 0.00 &-\\ 
  rs1937587 & 154060023 & 0.27 &-\\ 
  rs6921403 & 154063906 & 0.00 &-\\ 
  rs1937580 & 154076643 & 0.00 &+\\ 
  rs1937645 & 154082228 & 0.00 &+\\ 
  rs1892361 & 154099619 & 0.00 &-\\ 
  rs1937633 & 154104857 & 0.04 &-\\ 
  rs1937631 & 154105011 & 0.00 &-\\ 
  rs12527197 & 154107836 & 0.02 &+\\ 
  rs1892360 & 154111701 & 0.74 &-\\ 
  rs1892359 & 154112042 & 0.65 &-\\ 
  rs1892356 & 154112263 & 0.56 &+\\ 
  rs1937622 & 154113139 & 0.54 &-\\ 
  rs10485258 & 154113409 & 0.72 &-\\ 
  rs1937619 & 154114583 & 0.58 &-\\ 
  rs1748289 & 154121980 & 0.77 &-\\ 
  rs1781619 & 154135968 & 0.64 &-\\ 
  rs652051 & 154139344 & 0.74 &+\\ 
  rs10485262 & 154140199 & 0.69 &-\\ 
  rs9371312 & 154145492 & 0.81 &+\\ 
  rs1332849 & 154151117 & 0.48 &-\\ 
  rs9371749 & 154153369 & 0.28 &+\\ 
  rs9285539 & 154154532 & 0.08 &+\\ 
  rs9322439 & 154156250 & 0.07 &+\\ 
  rs11752884 & 154159710 & 0.25 &-\\ 
  rs4869813 & 154173845 & 0.13 &+\\ 
  rs4870241 & 154174963 & 0.00 &-\\ 
  rs9384156 & 154186720 & 0.13 &+\\ 
  rs2065139 & 154192175 & 0.89 &-\\ 
  rs689219 & 154198820 & 0.00 &-\\ 
  rs9371761 & 154202578 & 0.20 &-\\ 
  rs12199858 & 154204327 & 0.00 &+\\ 
  rs9371762 & 154213973 & 0.00 &-\\ 
  rs612450 & 154214357 & 0.00 &-\\ 
  rs9384159 & 154219177 & 0.00 &+\\ 
  rs6938958 & 154220427 & 0.00 &-\\ 
  rs581564 & 154221214 & 0.00 &+\\ 
  rs12202611 & 154237443 & 0.76 &-\\ 
  rs4870255 & 154237937 & 0.88 &-\\ 
   \hline
\end{tabular}
\caption{SNPs for OPRM1, chr6, position 154010496 - 154246867; $e$-value cutoff 0.225}
\end{table}

% latex table generated in R 3.3.2 by xtable 1.8-2 package
% Fri Oct 13 17:34:26 2017
\begin{table}[ht]
\centering
\begin{tabular}{lccc}
  \hline
SNP name & Location & $e$-value & Association \\ 
  \hline
rs10872828 & 133525348 & 0.72 &-\\ 
  rs9419702 & 133531153 & 0.09 &-\\ 
  rs7083395 & 133532269 & 0.77 &+\\ 
  rs9419624 & 133534822 & 0.06 &+\\ 
  rs7906770 & 133536902 & 0.28 &-\\ 
  rs9419569 & 133541881 & 0.06 &+\\ 
  rs9419629 & 133543210 & 0.06 &+\\ 
  rs7093241 & 133556596 & 0.72 &-\\ 
  rs9419649 & 133561098 & 0.91 &-\\ 
   \hline
\end{tabular}
\caption{SNPs for CYP2E1, chr10, position 133520406 - 133561220; $e$-value cutoff 0.72}
\end{table}

% latex table generated in R 3.3.2 by xtable 1.8-2 package
% Fri Oct 13 17:34:26 2017
\begin{table}[ht]
\centering
\begin{tabular}{lccc}
  \hline
SNP name & Location & $e$-value & Association \\ 
  \hline
rs7398343 & 111774068 & 0.34 &-\\ 
  rs7297186 & 111778178 & 0.36 &+\\ 
  rs3803167 & 111785586 & 0.00 &+\\ 
  rs10219736 & 111788402 & 0.00 &-\\ 
  rs16941437 & 111793039 & 0.00 &-\\ 
  rs3742004 & 111798553 & 0.75 &+\\ 
   \hline
\end{tabular}
\caption{SNPs for ALDH2, chr12, position 111766887 - 111817529; $e$-value cutoff 0.72}
\end{table}

% latex table generated in R 3.3.2 by xtable 1.8-2 package
% Fri Oct 13 17:34:26 2017
\begin{table}[H]
\centering
\begin{tabular}{lccc}
  \hline
SNP name & Location & $e$-value & Association \\ 
  \hline
rs4646312 & 19948337 & 0.41 & -\\ 
  rs165656 & 19948863 & 0.22 & -\\ 
  rs165722 & 19949013 & 0.24 &    +\\ 
  rs2239393 & 19950428 & 0.50 &+\\ 
  rs4680 & 19951271 & 0.60 &+\\ 
  rs4646316 & 19952132 & 0.81 &-\\ 
  rs165774 & 19952561 & 0.72 &-\\ 
  rs174699 & 19954458 & 0.07 &+\\ 
  rs165599 & 19956781 & 0.58 &-\\ 
  rs165728 & 19957023 & 0.02 &-\\ 
  rs165815 & 19959473 & 0.00 &+\\ 
  rs5993891 & 19959746 & 0.04 &-\\ 
  rs887199 & 19961955 & 0.04 &-\\ 
  rs2239395 & 19962203 & 0.07 &+\\ 
  rs2518824 & 19962963 & 0.59 &+\\ 
   \hline
\end{tabular}
\caption{SNPs for COMT, chr22, position 19941607 - 19969975; $e$-value cutoff 0.72}
\end{table}


% latex table generated in R 3.3.2 by xtable 1.8-2 package
% Fri Oct 13 17:34:26 2017
\begin{table}[H]
\centering
\begin{tabular}{lrrr}
  \hline
SNP name & Location & $e$-value & Association \\ 
  \hline
rs27072 & 1394522 & 0.87 &+\\ 
  rs40184 & 1395077 & 0.78 &-\\ 
  rs11564771 & 1398797 & 0.80 &-\\ 
  rs11133767 & 1401580 & 0.79 &+\\ 
  rs6869645 & 1404548 & 0.82 &+\\ 
  rs3776512 & 1407116 & 0.84 &+\\ 
  rs6347 & 1411412 & 0.83 &-\\ 
  rs27048 & 1412645 & 0.90 &-\\ 
  rs2042449 & 1416646 & 0.63 &+\\ 
  rs13161905 & 1417212 & 0.72 &-\\ 
  rs2735917 & 1420268 & 0.92 &+\\ 
  rs464049 & 1423905 & 0.21 &-\\ 
  rs460700 & 1429969 & 0.00 &-\\ 
  rs460000 & 1432825 & 0.00 &+\\ 
  rs4975646 & 1433401 & 0.88 &-\\ 
  rs403636 & 1438354 & 0.78 &-\\ 
  rs2617605 & 1442521 & 0.89 &+\\ 
  rs6350 & 1443199 & 0.93 &+\\ 
   \hline
\end{tabular}
\caption{SNPs for SLC6A3, chr5, position 1392790 - 1445430; $e$-value cutoff 0.72}
\end{table}

% latex table generated in R 3.3.2 by xtable 1.8-2 package
% Fri Oct 13 17:34:26 2017
\begin{table}[H]
\centering
\begin{tabular}{lccc}
  \hline
SNP name & Location & $e$-value & Association \\ 
  \hline
rs16967029 & 30195292 & 0.79 &+\\ 
  rs2051810 & 30195841 & 0.84 &-\\ 
  rs11658318 & 30206059 & 0.72 &-\\ 
  rs8079471 & 30218317 & 0.64 &+\\ 
  rs3760454 & 30222002 & 0.90 &+\\ 
   \hline
\end{tabular}
\caption{SNPs for SLC6A4, chr17, position 30194319 - 30236002; $e$-value cutoff 0.63}
\end{table}

% latex table generated in R 3.3.2 by xtable 1.8-2 package
% Fri Oct 13 17:34:26 2017
\begin{table}[H]
\centering
\begin{tabular}{lccc}
  \hline
SNP name & Location & $e$-value & Association \\ 
  \hline
rs2514229 & 113410000 & 0.87 &-\\ 
  rs11214654 & 113410917 & 0.86 &+\\ 
  rs7937641 & 113415976 & 0.63 &-\\ 
  rs12222458 & 113417603 & 0.73 &-\\ 
  rs10736470 & 113418371 & 0.73 &-\\ 
  rs12576506 & 113419869 & 0.85 &+\\ 
  rs10750025 & 113424042 & 0.66 &+\\ 
  rs7952106 & 113424558 & 0.70 &-\\ 
  rs4373974 & 113430486 & 0.88 &-\\ 
  rs4130345 & 113436487 & 0.88 &-\\ 
  rs7123697 & 113440331 & 0.78 &+\\ 
  rs6589386 & 113443753 & 0.75 &+\\ 
  rs4132966 & 113451589 & 0.86 &+\\ 
  rs7940164 & 113451765 & 0.90 &-\\ 
  rs4245155 & 113457324 & 0.92 &-\\ 
  rs11607834 & 113461680 & 0.92 &-\\ 
  rs12280220 & 113469219 & 0.93 &-\\ 
   \hline
\end{tabular}
\caption{SNPs for DRD2, chr11, position 113409595 - 113475691; $e$-value cutoff 0.63}
\end{table}


\section{Discussion on gene-specific findings in the MCTFR data}
\label{sec:DataSection}

{\it GABRA2:} As seen in the plots, the first two SNPs detected are close to two separate exons. The 4th and 5th detected SNPs, rs1808851 and rs279856, are at perfect LD with rs279858 in the larger 7188-individual dataset \citep{IronsThesis12}. This SNP had not been genotyped in our sample, but is the marker in GABRA2 that is most frequently associated in the literature with alcohol abuse \citep{CuiEtal12}. Interestingly, a single SNP RFGLS analysis of the same twin studies data that used Bonferroni correction on marginal $p$-values to detect SNPs had missed these SNPs \citep{IronsThesis12}. This highlights the advantage of our approach.

{\it ADH genes:} Multiple studies have associated rs1229984 in the ADH1B gene (position 99318162 of chromosome 4) with alcohol dependence (\url{https://www.snpedia.com/index.php/Rs1229984}), which as seen in the plot of ADH2 is close to an exon region. Our data does not contain this marker, but detects one SNP 20 kb upstream of this, rs17027523. Another SNP, rs3775540 at position 99304544 has an $e$-value of 0.226, so narrowly misses detection. This is close to rs1229984, and also rs1042026 at position 99307309, which \cite{MacgregorEtal08} found to be strongly associated with alcohol consumption.

The SNP rs17027523 is interesting: it resides in the uncharacterized long non-coding RNA gene LOC100507053. One previous study \citep{GelernterEtal14, XuEtal15} found significant associations for 5 SNPs in this gene with alcohol consumption for African American population through single-SNP analysis on non-familial GWAS data. Notably, their analysis found a much stronger evidence of the association in African-American part of the sample than the European American part, while our findings are entirely from a Caucasian sample.

{\it OPRM1:} Many of the SNPs analyzed in this gene have very low $e$-values, and tend to cluster together. The minor allele of the SNP rs1799971 (chr 6, position 154039662) has been associated with stronger alcohol cravings (\url{https://www.snpedia.com/index.php/Rs1799971}), and we detect rs12662873 at position 154040810.

{\it CYP2E1:} Five of the 9 SNPs studied are detected through our analysis. Four of them are within 10 kb of one another (base pairs 133534822 to 133543210 in chr 10). In the analysis of \cite{LindEtal12} rs4646976 at 133534223 position was most associated with a measure of breath alcohol concentration: this is within our detected region. This study had also detected rs4838767 in the promoter region of CYP2E1 (position 133520114) associated with multiple alcohol consumption measures. We detect rs9419702 at position 133531153.

{\it ALDH2:} All 6 SNPs we study are close to exons, and 5 get picked up by the $e$-value procedure. While all five are at a lesser base pair position than the well-known SNP rs671 ({\url{https://www.snpedia.com/index.php/Rs671}, position 111803962), one of the SNPs we analyze (rs16941437) is within 10 kb upstream of this SNP.

{\it COMT:} The SNP rs4680 has long been associated with schizophrenia and substance abuse, including alcoholism. A case-control study \citep{VoiseyEtal11} associated rs4680 and rs165774 with alcohol dependence through a SNP-wise chi-squared test, and had these two SNPs in high LD in their study population. Compared to this, in our simultaneous model of all COMT polymorphisms, the more well-known rs4680 has a below threshold $e$-value. 
% A previous case-control GWAS on unrelated individuals with European and African ancestry that also focused on analysis of well-studied genes also did not find any association between rs4680 and alcohol dependence \citep{OlfsonBierut08}.

{\it SLC6A3:} Our analysis does not detect rs27072, which has been associated with alcohol withdrawal symptoms (\url{https://www.snpedia.com/index.php/Rs27072}). 
%
%{\it SLC6A4:} The SNP rs1042173 has repeatedly been associated with alcohol consumption (\url{https://www.snpedia.com/index.php/Rs1042173}). In our analysis, the 3 SNPs closest to this have low $e$-values. One of them has $e$-value lower than the adaptive threshold $t=0.72$, while the other two narrowly miss it.
%
%{\it DRD2:} Five of the 7 SNPs analyzed have lower $e$-values than the rest, and all of them are in a 10 kb region, between positions 113415976 and 113424042 of chromosome 11. Two of these 5 have $e$-values below the gene-specific threshold. This region is within 3 kb upstream of rs1076560, which has multiple references of association with alcoholism (\url{https://www.snpedia.com/index.php/Rs1076560}). All the three DRD2 SNPs associated with alcoholism in a case-control study on an Eastern Indian study sample \citep{BhaskarEtal10}: rs2734835, rs1116313 and TaqID, are either inside or within 5 kb of this region. one of them (rs2734835) is also close to rs7937641, while the other two (rs1116313 and TaqID) are within 4 kb of another detected SNP in our analysis: rs10750025.

Finally, most $e$-values for the last 3 genes, i.e. SLC6A3, SLC6A4 and DRD2, are large: indicating weak SNP signals. We found this observation interesting, because variants of these genes have known interaction effects behind alcohol withdrawal-induced seizure \citep{KarpyakEtal10} and bipolar disorder \citep{WangEtal14}, as well as additive effect on the susceptibility to smoking addiction \citep{ErblichEtal05}.


% For this reason we also ran the $e$-values procedure on the combined set of SNPs from these genes, but did not detect any signal there as well for our sample. %change, look into rs1042173
 
%We detected two SNPs, rs10736470 and rs10750025, within 5 kb of one another. A previous case-control study identified three SNPs at high LD associated with alcoholism in Indian population: rs1116313, TaqID or rs1800498, and rs2734835. The first two of these are in between the two SNPs we detect.
\bibliographystyle{biometrics}
\bibliography{snpbib}

\end{document}