\section{Introduction}

Genome Wide Association Studies (GWAS) have identified a large number of genetic variants associated with complex diseases \citep{ChangEtal13,WheelerBarroso11}. The advent of economical high-throughput genotyping technology enables researchers to scan the genome with millions of SNPs, and improvements in computational efficiency in analysis techniques has facilitated parsing through this huge amount of data to detect significant associations \citep{VisscherEtal12}. Detecting small effects of individual SNPs requires large sample size \citep{ManolioEtal09}, which is a major challenge of these studies. For quantitative behavioral traits such as alcohol consumption, drug abuse, anorexia and depression, variation in genetic effects due to environmental heterogeneity brings in additional noise, further amplifying the issue. This is one of the motivations of performing GWAS on families instead of unrelated individuals, through which the environmental variation can be reduced \citep{BenyaminEtal09,MillerEtal12,VisscherEtal17}. However, association analysis using families can be computationally very challenging and thus single SNP association analysis is the standard tool for detecting SNPs. The Minnesota Twin Family Study \citep{MillerEtal12} with genome-wide data on identical twins, non-identical twins, biological offsprings, adoptees serve as the motivation for our methodology development in this paper.

Single-SNP analysis tests for association between the trait and a single SNP at a time. The SNPs with $p$-values lower than a particular threshold are considered to be associated with the trait. The GRAMMAR method of \cite{AulchenkoEtal07} and the association test of \cite{ChenAbecasis07} are examples of such techniques applied to familial data. While they can efficiently analyze GWAS data, they assume that phenotypic similarity within families is entirely due to their genetic similarity and ignore the effect of shared environment. As a result, they tend to lose power when analyzing data where shared environmental effects explain a substantial proportion of the total phenotypic variation (see \cite{McGueEtal13} and \cite{DeNeveEtal13} for example). In contrast, the RFGLS method proposed by \cite{LiEtal11} takes into account genetic and environmental sources of familial similarity and still provides fast inference through a rapid approximation of the single-SNP mixed effect model.

Single-SNP methods are prone to be less effective in detecting SNPs with weak signals \citep{ManolioEtal09}. This includes instances where multiple SNPs are jointly associated with the phenotype \citep{YangEtal12, Ke12, SchifanoEtal12}. Several methods of multi-SNP analysis have been proposed as alternatives. The kernel based association tests \citep{SchifanoEtal12, ChenEtal13, SchaidEtal13, ILazaEtal13} are prominent among such techniques. However, all of them test for whether a {\it group} of SNPs is associated with the phenotype being analyzed, and do not generally prioritize within the group and detect the individual SNPs primarily associated with the trait.

One way to solve this problem is to perform model selection. The methods of \cite{FrommeletEtal12} and \cite{ZhangEtal14} take this approach, and perform SNP selection from a multi-SNP model on GWAS data from \textit{unrelated individuals}. However, they rely on fitting models corresponding to multiple predictor sets, hence are computationally very intensive to implement in a linear mixed-effect framework for modeling familial data.

%Due to the weak signal of individual SNPs as well as the heavy amount of correlation among them, detecting SNPs that are actually associated with the quantitative trait being analyzed is statistically challenging. A major impediment of estimating effects of multiple SNPs \textit{while} taking into account theirIn any kind of GWAS, fitting separate models on single markers typically suffer from loss of power.
% However the dependent data structure and large sample sizes in familial GWAS data calls for usage of suitable statistical models, for example mixed effet modelling, which makes even training a single model computationally intensive. and because of this any traditional variable selection approach is infeasible in such setup.

In this paper we propose a fast and scalable model selection technique thats fits a single model to a family data, and aims to identify important genetic variants with weak signals through joint modelling of multiple variants. We consider only main effects of the variants, but this can be extended to include higher-order interactions. We achieve this by extending the recently proposed framework of $e$-values \citep{MajumdarChatterjee17}. For any estimation method that provides consistent estimates (at a certain rate relative to the sample size) of the vector of parameters, $e$-values quantify the proximity of the sampling distribution for a restricted parameter estimate to that of the full model estimate in a regression-like setup. A variable selection algorithm using the $e$-values has the following simple and generic steps:
%
\begin{enumerate}
\item Fit the full model, i.e. where all predictor effects are being estimated from the data, and use resampling to estimate its $e$-value;

\item Set an element of the full model coefficient estimate to 0 and get an $e$-value for that predictor using resampling distribution of previously estimated parameters- repeat this for all predictors;

\item Select predictors that have $e$-values below a pre-determined threshold.
\end{enumerate}

The above algorithm offers multiple important benefits in the SNP selection scenario. Unlike other model selection methods, only the full model needs to be computed here. It thus offers the user more flexibility in utilizing a suitable method of estimation for the full model. Our method allows for fitting multi-SNP models, thereby accommodating cases of modelling multiple correlated SNPs or closely located multiple causal SNPs simultaneously. Finally, we use the Generalized Bootstrap \citep{ChatterjeeBose05} as our chosen resampling technique. Instead of fitting a separate model for each bootstrap sample, it computes bootstrap estimates using Monte-Carlo samples from the resampling distribution as weights, and reusing model objects obtained from the full model. Consequently, the resampling step becomes very fast and parallelizable.

The rest of the paper is organized as follows. Section \ref{sec:modelSection} provides background information on the GWAS Family dataset we use in our case study, as well as introduces the statistical framework we use to model this data. We start Section \ref{sec:methodsSection} by providing a technical introduction to the $e$-values framework, then elaborate on the necessary modifications for adapting it to our modelling scenario, and present details of the bootstrap procedure. We illustrate the performance of this method on synthetic datasets in Section~\ref{sec:SimSection}. In Section \ref{sec:DataSection} we analyze our GWAS dataset using the $e$-values technique to identify novel SNPs from multiple genes that have been reported to influence alcohol consumption in individuals. Finally in Section \ref{sec:endSection} we outline directions of future research. We include the proofs of all new results stated, specifically, theorems \ref{thm:quantileThm} and \ref{thm:bootThm}, in the supplementary material.