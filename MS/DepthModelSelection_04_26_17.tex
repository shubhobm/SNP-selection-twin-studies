%\documentclass[fleqn,11pt]{article}
\documentclass[aos]{imsart}

\usepackage{mycommands,amssymb,amsmath,mathabx,amsthm,color,pagesize,outlines,cite,subfigure}
\usepackage[small]{caption}
%\usepackage{epsfig}
\usepackage{graphicx}

\usepackage[round]{natbib}

% for algorithm
\usepackage[noend]{algpseudocode}
\usepackage{algorithm}
\usepackage{hyperref} % for linking references 
\usepackage{stackrel}

%\addtolength{\evensidemargin}{-.5in}
%\addtolength{\oddsidemargin}{-.5in}
%\addtolength{\textwidth}{0.9in}
%\addtolength{\textheight}{0.9in}
%\addtolength{\topmargin}{-.4in}

%% measurements for 1 inch margin
%\addtolength{\oddsidemargin}{-.875in}
%\addtolength{\evensidemargin}{-.875in}
%\addtolength{\textwidth}{1.75in}
%\addtolength{\topmargin}{-.875in}
%\addtolength{\textheight}{1.75in}
%\usepackage{setspace}
%\doublespacing

%\pagestyle{myheadings}
%\markboth{}{\underline{{\bf Notes: (do not circulate)} \hspace{4.5cm} {\sc  Ansu Chatterjee} \hspace{0.25cm}}}

%% Appendix theorem counter
\usepackage{chngcntr}
\usepackage{apptools}
\AtAppendix{\counterwithin{Theorem}{section}}

\numberwithin{equation}{section}

%\DeclareMathOperator*{\vec}{vec}
\DeclareMathOperator*{\diag}{diag }
\DeclareMathOperator*{\Tr}{Tr}
\DeclareMathOperator*{\argmin}{arg\,min}
\DeclareMathOperator*{\argmax}{arg\,max}


\begin{document}

\newtheorem{Theorem}{Theorem}[section]
\newtheorem{Lemma}[Theorem]{Lemma}
\newtheorem{Corollary}[Theorem]{Corollary}
\newtheorem{Proposition}[Theorem]{Proposition}
\newtheorem{Conjecture}[Theorem]{Conjecture}
\theoremstyle{definition} \newtheorem{Definition}[Theorem]{Definition}
\theoremstyle{definition} \newtheorem{Example}[Theorem]{Example}

%\title{Generalized Model Discovery using Statistical Evaluation Maps}
%\date{}
%\author{Subhabrata Majumdar, Snigdhansu Chatterjee}
%\maketitle

\begin{frontmatter}
\title{Generalized Model Discovery using Statistical Evaluation Maps\thanksref{T1}}
\runtitle{Generalized Model Discovery with $e$-values}
%\thankstext{T1}{Footnote to the title with the ``thankstext'' command.}

\begin{aug}
\author{\fnms{Subhabrata} \snm{Majumdar}\thanksref{t1,t2,m1}\ead[label=e1]{majum010@umn.edu}}
\and
\author{\fnms{Snigdhansu} \snm{Chatterjee}\thanksref{t3,m1,m2}\ead[label=e2]{chatt019@umn.edu}}
%\and
%\author{\fnms{Third} \snm{Author}\thanksref{t1,m2}
%\ead[label=e3]{third@somewhere.com}
%\ead[label=u1,url]{http://www.foo.com}}

\thankstext{t1}{aaa}
\thankstext{t2}{First supporter of the project}
\thankstext{t3}{Second supporter of the project}
\runauthor{Majumdar and Chatterjee}

\affiliation{University of Minnesota Twin Cities}

\address{Ford Hall,\\
224, Church Street SE\\
Minneapolis, MN 55414, USA\\
\printead{e1}\\
\phantom{E-mail:\ }\printead*{e2}}

%\address{Address of the Third author\\
%Usually a few lines long\\
%Usually a few lines long\\
%\printead{e3}\\
%\printead{u1}}
\end{aug}

\begin{abstract}
The abstract should summarize the contents of the paper.
It should be clear, descriptive, self-explanatory and not longer
than 200 words. It should also be suitable for publication in
abstracting services. Please avoid using math formulas as much as possible.

This is a sample input file.  Comparing it with the output it
generates can show you how to produce a simple document of
your own.
\end{abstract}

\begin{keyword}[class=MSC]
\kwd[Primary ]{60K35}
\kwd{60K35}
\kwd[; secondary ]{60K35}
\end{keyword}

\begin{keyword}
\kwd{sample}
\kwd{\LaTeXe}
\end{keyword}

\end{frontmatter}

%\begin{abstract}
%We introduce a one-step model selection technique for general regression estimators
%to provide a solution to the problem of statistical model selection. Under very general assumptions,
%this technique correctly identifies the set of non-zero values in the true coefficient (of length p) by
%comparing only p + 1 models. We start by defining our selection criterion for a class of candidate
%models larger than considered before, and providing population-level results that differentiate between
%correct and wrong models within this class. After this we provide results for a general bootstrap
%scheme to estimate the criterion in a sample setup, and discuss its details for linear and linear mixed
%models. Simulations and a real data example demonstrate the efficacy of our method over existing
%model selection strategies in terms of detecting the correct set of predictors as well as accurate out-ofsample predictions. At the end we also discuss some immediate applications and possible extensions
%of this foundational methodology
%\end{abstract}

%\section{Introduction}
\input{Sections/Introduction_02_05_17_1}
\input{Sections/Frame_of_models}
\input{Sections/StatisticalEvaluationMaps_and_evalues}
\input{Sections/ResamplingSection}
\input{Sections/FastVariableSelection}

%\section{Simulation studies}
%\section{Application: Linear mixed effect model for Indian Monsoon precipitation}
%\section{Caveats, conclusions}

\input{Sections/Simulation}
\input{Sections/IndianMonsoon}
\input{Sections/CaveatsConclusion}
\input{Sections/Proofs}

%\input{./Sections/ParametersScientificModels_01_28_17}

%\input{./Sections/FrameOfModels_02_05_17_1}
%
%\input{./Sections/TransformationToCommonPlatform_02_05_17_1}
%
%\bredbf There are two parallel versions of the next section, combine
%\_ 1 and \_3
%\eredbf
%\input{./Sections/StatisticalEvaluationMaps_02_05_17_1}

% In using BibteX, use wb_stat.bst
%\bibliographystyle{wb_stat}
%\bibliography{bibfile}

\newpage 



\bibliographystyle{apalike}
\bibliography{MSbib,climate}
\end{document}